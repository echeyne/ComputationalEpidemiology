
% !TEX TS-program = pdflatex
% !TEX encoding = UTF-8 Unicode

% This is a simple template for a LaTeX document using the "article" class.
% See "book", "report", "letter" for other types of document.

\documentclass[12pt, oneside, openany]{article} % use larger type; default would be 10pt

\usepackage[utf8]{inputenc} % set input encoding (not needed with XeLaTeX)

%%% Examples of Article customizations
% These packages are optional, depending whether you want the features they provide.
% See the LaTeX Companion or other references for full information.

%%% PAGE DIMENSIONS
\usepackage{geometry} % to change the page dimensions
\geometry{letterpaper} % or letterpaper (US) or a5paper or....
% \geometry{margin=2in} % for example, change the margins to 2 inches all round
% \geometry{landscape} % set up the page for landscape
%   read geometry.pdf for detailed page layout information

\usepackage{graphicx} % support the \includegraphics command and options

\usepackage[titletoc,toc,title]{appendix}

% \usepackage[parfill]{parskip} % Activate to begin paragraphs with an empty line rather than an indent

%%% PACKAGES
\usepackage{booktabs} % for much better looking tables
\usepackage{array} % for better arrays (eg matrices) in maths
\usepackage{paralist} % very flexible & customisable lists (eg. enumerate/itemize, etc.)
\usepackage{verbatim} % adds environment for commenting out blocks of text & for better verbatim
\usepackage{subfig} % make it possible to include more than one captioned figure/table in a single float
%\usepackage{forest} % Used for creating tree diagrams
\usepackage{color}
\usepackage{longtable}
\usepackage{amsmath}
% These packages are all incorporated in the memoir class to one degree or another...

%%% HEADERS & FOOTERS
\usepackage{fancyhdr} % This should be set AFTER setting up the page geometry
\pagestyle{fancy} % options: empty , plain , fancy
\renewcommand{\headrulewidth}{0pt} % customise the layout...
\lhead{}\chead{}\rhead{}
\lfoot{}\cfoot{\thepage}\rfoot{}

%%% SECTION TITLE APPEARANCE
\usepackage{sectsty}
%\allsectionsfont{\sffamily\mdseries\upshape} % (See the fntguide.pdf for font help)
% (This matches ConTeXt defaults)

%%% ToC (table of contents) APPEARANCE
\usepackage[nottoc,notlof,notlot]{tocbibind} % Put the bibliography in the ToC
\usepackage[titles,subfigure]{tocloft} % Alter the style of the Table of Contents
\renewcommand{\cftsecfont}{\rmfamily\mdseries\upshape}
\renewcommand{\cftsecpagefont}{\rmfamily\mdseries\upshape} % No bold!

\usepackage{listings}
%\usepackage{minted}

\newcommand{\deriv}[2]{\frac{\mathrm d #1}{\mathrm d #2}}
\newcommand{\pderiv}[2]{\frac{\partial #1}{\partial #2}}

\newcommand\Tstrut{\rule{0pt}{2.6ex}}       % "top" strut
\newcommand\Bstrut{\rule[-0.9ex]{0pt}{0pt}} % "bottom" strut
\newcommand{\TBstrut}{\Tstrut\Bstrut}

\setlength{\footskip}{1.5cm}

%%% END Article customizations

\hyphenation{releaseNameAddressPhoneToThirdParty}

\begin{document}
\begin{titlepage}
\begin{center}
\vspace*{150px}
\LARGE{\textbf{An Examination of Networks in Computational Epidemiology} \\}
\vspace{12px}
\Large{Final Report\\}
\vspace{12px}
\large{Anthony Culos, Zach Holland, and Emily Millard \\}
\vspace{12px}
\large{December 11, 2015 \\}
\end{center}
\end{titlepage}
\tableofcontents
\newpage

\section{Introduction}
This paper aims to use computational epidemiology as a vehicle to analyze the diffusion of disease across a large network. Global trends of increasingly dense urbanization, international travel, and an aging population place substantial importance on our ability to understand, predict, and prevent the spread of disease throughout a network. Due to the sensitivity of medical data it was not possible for this paper to analyze the diffusion of disease across an infectious disease network but in the paper ``The Structure of Online Diffusion Networks" Goel \emph{et al.} established the adoption of new ideas, norms, and behaviours across an interpersonal network closely resembles the spread of infectious disease throughout a network. Using the data set ``The Anatomy of a Scientific Rumor" by Domenico \emph{et al.} this paper will focus on the spread of the Higgs boson particle discovery across Twitter. Specifically, this paper will perform two separate analysis. First it will look at the general diffusion of knowledge about the Higgs boson particle across Twitter and then analyze the diffusion of knowledge in terms of the community structure of the contact network. Finally, after each analysis is performed a general summary and relation to typical diffusion networks will follow.

%%% Article Summary %%%
\section{Overview of \textit{Computational Epidemiology} as it Relates to our Project}
An epidemic can be characterized as a widespread occurrence of an illness or other health-related event at a particular time; a pandemic is an epidemic of worldwide proportions. Due to emerging global trends of increasingly dense urbanization, more local and global travel, and a generally older population, controlling and responding to future epidemics and possibly pandemics will become significantly more challenging. However, advances in computing, big data, and computational thinking have created new opportunities to control and prevent the spread of infectious disease. Computation plays a pivotal role in supporting real-time epidemiology because using controlled experiments to understand the spread of disease is much harder to perform than computer modeling, and often experiments are impossible due to ethical constraints. 

In 1760 Daniel Bernoulli developed the first mathematical model in epidemiology. Using his model, Bernoulli established that vaccination could help increase the life expectancy in the French population. Presently, the simplest and most commonly used model to demonstrate epidemic processes is the \emph{SIR} model. In this model a population of size $N$ is divided into three states: susceptible ($S$), infective ($I$), and removed or recovered ($R$). Each infected person can infect any susceptible person (independently) with probability $\beta$ and can recover with probability $\gamma$. Let $S(t)$, $I(t)$, and $R(t)$ represent the number of people that are in either the susceptible, infected, or removed states at a given time $t$. Furthermore let $s(t) = \frac{S(t)}{N}$, $i(t) = \frac{I(t)}{N}$, and $r(t) = \frac{R(t)}{N}$. (i.e. the fractions of the population in each of the states.) And it follows that: $s(t) + i(t) + r(t) = 1$. 

Using the complete mixing requirement which assumes that every person is in contact with everyone in the population, the following system of differential equations describe the dynamics of the SIR model:

\begin{align}
	\deriv{s(t)}{t} &= -\beta s(t) i(t) \\
	\deriv{i(t)}{t} &= \beta s(t) i(t) - \gamma i(t) \\
	\deriv{r(t)}{t} &= \gamma i(t)
\end{align}

A common use for the SIR model is to determine the likelihood of an epidemic infecting a large fraction of the population. According to the SIR model, a large scale infection will occur if and only if $R_{0} = \frac{\beta}{\gamma} > 0 $, where $R_{0}$ is the \textit{reproductive number}. Public health decision making is often very concerned with controlling $R_{0}$.

The SIR model makes many assumptions, the most significant of which is the complete mixing requirement which is often unrealistic. A more recent approach to understand the spread of an epidemic is \textit{networked epidemiology}. Networked epidemiology seeks to understand the interactions between three distinct components of epidemiology: individual behaviors of agents; unstructured, heterogeneous multi-scale networks; and the dynamical processes on these networks. This type of analysis is focused on the belief that a better understanding of the underlying network and individual behaviors within the network will provide better insights into contagion dynamics and response strategies. 

To apply this network model, let $G(V, E)$ denote a contact graph on a population, where $V$ is the set of people in the population and $E$ is the set of contact pairs in the population (i.e. $\{u,v\}\in E~ \mathrm{iff}~u,v\in V~\mathrm{and}~u,v$ have come into contact with each other). Additionally, let $N(v)$ denote a set of neighbors of $v$. Applying the SIR model to $G$, a node is in either a $S$, $I$, or $R$ state. For each $\{u,v\}$ pair, infection can potentially spread along edge $e = \{u, v\}$ with probability of $ \beta(e,t) $ at time $t$ after $u$ becomes infected. This probability is conditional on node $v$ remaining susceptible until time $t$.~\cite{marathe} 

\subsection{Technology Diffusion as an Analogue to Disease Transmission}
It has been well studied that the adoption of new ideas, norms, behaviors, and products through an interpersonal network closely resembles the spread of infectious disease through an affected population~\cite{goel}. Due to the difficulty of finding proper examples of population contact networks and their associated disease transmission networks, we propose to use interpersonal networks and the diffusion of technology adoption on those networks as an analogue.

%%% Data Analysed %%%
\section{Data Analysed}
We were able to find a comprehensive data set of information diffusion from ``The Anatomy of a Scientific Rumor'' by Domenico \emph{et al}.~\cite{domenico}. Domenico and his team tracked the process of information diffusion across Twitter after the announcement of the the discovery of a particle displaying the properties of the Higgs boson on July 4th, 2012. The researchers made their data available through the Stanford Network Analysis Project (SNAP)~\cite{snap}, a repository of large network datasets. For this report we analyzed two networks: the contact network in which the information of the Higgs discovery spread and the greater friends/followers network of all Twitter users involved in the information diffusion. The diffusion network has Twitter users as nodes, and interactions such as mentions, replies, and retweets as edges. Each edge represents a direction of infection between users. The larger contact network has users as nodes and the edges represent the relationship between the user and its direction (i.e. following, follower). It is important to note that a single interaction can incur multiple edges from a node. For instance if a user mentions another in a tweet that gets retweeted both a mention and a retweet edge will be produced going from the retweeting account to each other user respectively.

\section{General Diffusion Analysis}
\subsection{Strategy}
The diffusion of knowledge about the Higgs boson particle across Twitter will be analyzed in intervals of 300 seconds as opposed to analyzing the diffusion over each second. This simplification will make computation easier because it reduces the number of data points from 563069 to 2016. Our diffusion analysis will consist of two different datasets constructed from the original data provided by SNAP. The first will be the sum of all infection events over a given time interval paired with the amount of infected nodes at a given time interval and the second will be the sum of all infection events with respect to mentions, replies, and retweets. Our analysis will be focused on the first dataset and then compared to the second dataset. After our initial comparison we will measure our results to see whether they agrees with the literature of diffusion on large networks. According to a Goel \emph{et al}. the presence of significant cascades in large networks are extremely rare. If they occur the majority of infections occur within one degree of the seed~\cite{goel}. That is, we can expect to see a large infection event occur and then the and then the infection should very quickly die off.

\subsection{Analysis}
Present literature on diffusion in large networks states that the network should be inactive until a large cascade of infections occurs. The cascade will start quickly and will die out very soon after it started. Our analysis of the SNAP data agrees with these findings. Once establishing this agreement we felt it important to then look at the infections prior to and after the cascade. Information on the infections per time period are given by Table~\ref{tbl:stats}, and this information is visualized in figures~\ref{fig:infection-count},~\ref{fig:newly-infected}, and~\ref{fig:overlay}.

\begin{table}
\centering
\begin{tabular}{| c || c | c | c | c |}
	\hline
   		& Standard Dev. & Mean & Max & Variance \\
   	\hline\hline
  	New Infections & 544.785 & 278.301 & 4765 & 296643.668 \\
	\hline
\end{tabular}
  \caption{Some general statistics of the rates of infection for each time period.}
  \label{tbl:stats}
\end{table}

\begin{figure}
\centering
    \includegraphics{infection-count.png}
    \caption{The cumulative number of infections for all the time intervals.}
    \label{fig:infection-count}
\end{figure}

\begin{figure}
\centering
    \includegraphics{newly-infected.png}
    \caption{The number of new infections during each time interval.}
    \label{fig:newly-infected}
\end{figure}

If we overlay the two plots from Figures~\ref{fig:infection-count} and~\ref{fig:newly-infected} (as is shown in Figure~\ref{fig:overlay}) it becomes obvious that the majority of infections happens within a very short amount of time relative to the total time of the diffusion.

\begin{figure}
\centering
    \includegraphics{overlay.png}
    \caption{The overlay of the two plots from figures~\ref{fig:infection-count} and~\ref{fig:newly-infected} shows the relation between number of new infections and the total number of infections at a given interval.}
    \label{fig:overlay}
\end{figure}

Between the times 1341380272-1341430372 (time since the UNIX Epoch in seconds) 298571 infections occur out of a total 563069 infections over the lifetime of the diffusion. This 835 min (approximately 14 hour) period accounts for over 50\% of all the infections. This agrees with our predictions that the information will both start and end quickly after the initial cascade event. The growth and decay of infections is even more evident if we split our data before and after the peak infection time as is shown in Figure~\ref{fig:before-after}.

\begin{figure}
\centering
    \includegraphics{before.png}
      \includegraphics{after.png}
    \caption{Display the number of new infections before and after the peak infection. This shows very well how rapidly information diffuses and then decays on large networks.}
    \label{fig:before-after}
\end{figure}

Once we established a general understanding of how the diffusion of the Higgs boson announcement spreads on a large network, we began analyzing the specific avenues of diffusion available (replies, mentions, and retweets) for any anomalies.

\begin{table}
\centering
\begin{tabular}{| c || c | c | c | c |}
	\hline
   		& Standard Dev. & Mean & Variance &Total \\
   	\hline\hline
  	Replies & 32.342 & 18.305 & 1045.999 & 36902 \\
  	\hline
  	Mentions & 152.613 & 84.939 & 23290.67 & 171237 \\
  	\hline
  	Retweets & 362.454 & 176.057 & 131373.1 & 354930 \\
	\hline
\end{tabular}
  \caption{Statistics of each type of diffusion observed in the SNAP Higgs boson data set.}
  \label{tbl:type-stats}
\end{table}

\begin{figure}
\centering
    \includegraphics{methods.png}
    \caption{Shows how the different means of diffusion relate to each other.}
    \label{fig:methods}
\end{figure}

Table~\ref{tbl:type-stats} and Figure~\ref{fig:methods} show the rate of infection of each method of diffusion. These representations follows closely to the aggregate diffusion data with no deviation from what we expect to see. It is interesting to note the popularity of the methods used to spread the Higgs announcement. The most popular method being the one with the least involvement (retweeting) and the most involved method of interaction (reply) is the least popular. This is most likely due to the majority of people not being motivated enough about the discovery to start a conversation but enough interest in the subject to want to spread the news.

\subsection{Conclusion}
Our analysis showed the Higgs boson discovery diffused on the Twitter network in a manner to how data diffuses across other large networks. The diffusion of information began with little activity and then a large infection event that initialized and decayed at a rapid rate. This quick start and decay accounted for majority of the infections throughout the diffusion of the Higgs boson discovery across the Twitter network. These properties are consistent with the expected behaviour of a diffusion in a large network therefore we can conclude that the discovery of the Higgs boson follows the expected behaviour of diffusion.

\section{Diffusion Analysis In Terms of the Community Structure of the Contact Network}
\subsection{Community Structure of Networks}
In complex networks groups of nodes that are more likely to be connected to each other than to those of other groups are called \emph{communities}. Understanding the community organization of a network provides insight into the fundamental structure of the network and can be used to predict the behaviours or actions of the nodes. One way of defining a community in a network is to identify smaller networks within the larger network that have a specific structure. One type of structure that is of interest, proposed by J. Nastos and Y. Gao in 2013~\cite{nastos:2013}, is quasi-threshold graphs. Quasi-threshold graphs have been blah blah.

There are a variety of equivalent definitions for quasi-threshold graphs, but the simplest is stated in terms of forbidden induced subgraphs:. A qualitative description. Each connected component will have a universal vertex ... 

We are interested to see how the twitter data diffuses through the community structure of the contact network. blah blah what we expect.

\subsection{Diffusion Through Communities}
For our analysis we used a recently proposed algorithm~\cite{brandes} to edit the Higgs Twitter contact network (the network of friends and followers) to approximately the nearest quasi-threshold graph. Even though this editing algorithm runs in polynomial time, our implementation would not scale to a network of more than about 6,000 nodes with the time and computational resources that we had. The Higgs Twitter contact network has hundreds of thousands of nodes and many more edges so we were not able to edit it directly. Instead, we built a section of the network by examining the series of nodes that became infected from the beginning of the data up until we had a collection of 6,000 nodes. We then built a graph using those 6,000 nodes, connecting two nodes with an edge if there was an edge in the original Higgs Twitter contact network. This resulted in a more managable section of the larger network to work with.

After editing our 6,000 node network to its quasi-threshold community structure, we applied the infection (retweet, mention, reply) data to the communities in a series of 20 time steps, with time step 0 having no nodes infected, and time step 20 having all the nodes infected. Each time step was 6,632 seconds long (about 1.8 hours), which gives us a total snapshot of the diffusion that is 132,658 seconds (about 37 hours). We treated retweets, mentions, and replys all as equal infections and coloured them red. The diffusion of the infection through the small network after 1 time step is shown in Figure~\ref{fig:net-time1}. A middle step of the diffusion is shown in Figure~\ref{fig:net-time11}, and the penultimate step before all the nodes are infected is show in Figure~\ref{fig:net-time19}. All the graph visualizations were produced with yEd.

\begin{figure}
\centering
    \includegraphics[scale=0.25]{step1.png}
    \caption{The diffusion of the infection through the identified communities after 1 time step. Infected nodes are colored in red. Communities with less than 30 nodes have been ommited for the sake of compactness.}
    \label{fig:net-time1}
\end{figure}

\begin{figure}
\centering
    \includegraphics[scale=0.25]{step11.png}
    \caption{The diffusion of the infection through the identified communities after 11 time steps.}
    \label{fig:net-time11}
\end{figure}

\begin{figure}
\centering
    \includegraphics[scale=0.25]{step19.png}
    \caption{The diffusion of the infection through the identified communities after 19 time steps.}
    \label{fig:net-time19}
\end{figure}

From Figures \ref{fig:net-time1}, \ref{fig:net-time11}, \ref{fig:net-time19} we see can that nodes in each the communities become infected roughly the same rate. This is different from what we had hypothesized, the communities did not remain relatively isolated from the infection and diffuse blah blah.

In Table~\ref{tbl:series} we have visualized the third largest connected component at a series of consecutive time steps. This component consists of 364 nodes and 533 edges. We found that much of the spread of the infection in this component happens in a short amount of time. For the first 8 time steps the component only has a few infected nodes and it is between steps 9 and 11 that the majority of the infection occurs with the remaining 9 steps slowly infecting the remainder.

We noticed that between time steps 8 and 9 the root node of the community becomes infected and the following steps are when the diffusion cascades through the rest of the nodes. This might imply that the key node for infecting the community is the root node. The other large connected components behave similarly.

\begin{table}[p]
\centering
\vspace{-1cm}
\begin{tabular}{| c | c |}
	\hline
	&\\
   	\includegraphics[scale=0.11]{comp6.png} & \includegraphics[scale=0.11]{comp7.png} \\
	Step 6 & Step 7 \Bstrut \\
   	\hline
&\\
  	\includegraphics[scale=0.11]{comp8.png} & \includegraphics[scale=0.11]{comp9.png} \\
	Step 8 & Step 9 \Bstrut \\
  	\hline
&\\
  	\includegraphics[scale=0.11]{comp10.png} & \includegraphics[scale=0.11]{comp11.png}  \\
	Step 10 & Step 11 \Bstrut \\
  	\hline
&\\
  	\includegraphics[scale=0.11]{comp12.png} & \includegraphics[scale=0.11]{comp13.png}  \\
	Step 12 & Step 13 \Bstrut \\
	\hline
\end{tabular}
  \caption{The state of infection of the 3rd largest connected component at a series of time steps. The first 5, and the last 7 steps are not shown.}
  \label{tbl:series}
\end{table}

\subsection{Conclusion}
From our preliminary investigation we can see that all of the communities become infected at roughly the same rate, but it appears that the root nodes of the quasi-threshold communities are important to spreading the infection to the rest of the community. This is most likely due to the fact that the root is connected to every other node in the community and has the most influence on the other nodes. To help stop the spread of an epidemic it would be important to identify and isolate the people who are the root nodes in their communities. Conversley, if you want to spread something like information quickly through a community, you should release it to the person who is the root. 

It is difficult to draw firm conclusions on the behaviour of diffusion through the quasi-threshold communities of the network without looking at a larger portion of the Higgs Twitter dataset. This analysis was also mostly qualitative and warrants further and more precise investigation on this data set and as well as other diffusion data sets. Additionally, we did not examine the effect the time of day or the geographic locations of the users represented by each node had on the rate of diffusion. These both may have an effect on the way that the infection spreads.

\newpage
\begin{thebibliography}{1}
 \bibitem{marathe} 
 M. Marathe and A. Vullikanti. ``Computational Epidemiology.'' \emph{Communications of the ACM}, vol. 56, no. 7, pp. 88-96, 2013.
 \bibitem{goel} 
 S. Goel \emph{et al}. ``The Structure of Online Diffusion Networks,'' in \emph{Proceedings of the 13th ACM Conference on Electronic Commerce}. New York, NY: ACM. 2012. pp. 623-638.
 \bibitem{domenico}
 D. Domenico \emph{et al}. ``The Anatomy of a Scientific Rumor.'' \emph{Scientific Reports}, vol. 3, no. 2980. Oct. 18, 2013.
\bibitem{snap}
Stanford Network Analysis Project. ``Higgs Twitter Dataset,'' Accessed Nov. 2015, https://snap.stanford.edu/data/higgs-twitter.html
\bibitem{nastos:2013}
J. Nastos and Y. Gao, ``Familial groups in social networks,'' \emph{Soc. Networks}, vol. 35, no. 3, pp. 439-450, July 2013.
\bibitem{brandes}
U. Brandes \emph{et al.}, ``Fast Quasi-Threshold Editing,'' \emph{Cornell University Library,} arXiv:1504.07379, Apr. 28, 2015
 \end{thebibliography}

\end{document}